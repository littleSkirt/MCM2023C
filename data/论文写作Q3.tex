% Options for packages loaded elsewhere
\PassOptionsToPackage{unicode}{hyperref}
\PassOptionsToPackage{hyphens}{url}
%
\documentclass[
]{article}
\usepackage{amsmath,amssymb}
\usepackage{iftex}
\ifPDFTeX
  \usepackage[T1]{fontenc}
  \usepackage[utf8]{inputenc}
  \usepackage{textcomp} % provide euro and other symbols
\else % if luatex or xetex
  \usepackage{unicode-math} % this also loads fontspec
  \defaultfontfeatures{Scale=MatchLowercase}
  \defaultfontfeatures[\rmfamily]{Ligatures=TeX,Scale=1}
\fi
\usepackage{lmodern}
\ifPDFTeX\else
  % xetex/luatex font selection
\fi
% Use upquote if available, for straight quotes in verbatim environments
\IfFileExists{upquote.sty}{\usepackage{upquote}}{}
\IfFileExists{microtype.sty}{% use microtype if available
  \usepackage[]{microtype}
  \UseMicrotypeSet[protrusion]{basicmath} % disable protrusion for tt fonts
}{}
\makeatletter
\@ifundefined{KOMAClassName}{% if non-KOMA class
  \IfFileExists{parskip.sty}{%
    \usepackage{parskip}
  }{% else
    \setlength{\parindent}{0pt}
    \setlength{\parskip}{6pt plus 2pt minus 1pt}}
}{% if KOMA class
  \KOMAoptions{parskip=half}}
\makeatother
\usepackage{xcolor}
\usepackage{longtable,booktabs,array}
\usepackage{calc} % for calculating minipage widths
% Correct order of tables after \paragraph or \subparagraph
\usepackage{etoolbox}
\makeatletter
\patchcmd\longtable{\par}{\if@noskipsec\mbox{}\fi\par}{}{}
\makeatother
% Allow footnotes in longtable head/foot
\IfFileExists{footnotehyper.sty}{\usepackage{footnotehyper}}{\usepackage{footnote}}
\makesavenoteenv{longtable}
\usepackage{graphicx}
\makeatletter
\def\maxwidth{\ifdim\Gin@nat@width>\linewidth\linewidth\else\Gin@nat@width\fi}
\def\maxheight{\ifdim\Gin@nat@height>\textheight\textheight\else\Gin@nat@height\fi}
\makeatother
% Scale images if necessary, so that they will not overflow the page
% margins by default, and it is still possible to overwrite the defaults
% using explicit options in \includegraphics[width, height, ...]{}
\setkeys{Gin}{width=\maxwidth,height=\maxheight,keepaspectratio}
% Set default figure placement to htbp
\makeatletter
\def\fps@figure{htbp}
\makeatother
\setlength{\emergencystretch}{3em} % prevent overfull lines
\providecommand{\tightlist}{%
  \setlength{\itemsep}{0pt}\setlength{\parskip}{0pt}}
\setcounter{secnumdepth}{-\maxdimen} % remove section numbering
\ifLuaTeX
  \usepackage{selnolig}  % disable illegal ligatures
\fi
\IfFileExists{bookmark.sty}{\usepackage{bookmark}}{\usepackage{hyperref}}
\IfFileExists{xurl.sty}{\usepackage{xurl}}{} % add URL line breaks if available
\urlstyle{same}
\hypersetup{
  hidelinks,
  pdfcreator={LaTeX via pandoc}}

\author{}
\date{}

\begin{document}

\hypertarget{ux8bbaux6587ux5199ux4f5c}{%
\subsection{论文写作}\label{ux8bbaux6587ux5199ux4f5c}}

\hypertarget{q3}{%
\subsubsection{Q3}\label{q3}}

Develop and summarize a model to classify solution words by difficulty.
Identify the attributes of a given word that are associated with each
classification. Using your model, how difficult is the word EERIE?
Discuss the accuracy of your classification model.

开发并总结了一个模型,按难度对解词进行分类。识别与每个分类相关联的给定单词的属性。用你的模型,EERIE这个词有多难?讨论你的分类模型的准确性。

\hypertarget{ux95eeux9898ux5206ux6790}{%
\paragraph{问题分析}\label{ux95eeux9898ux5206ux6790}}

There are four tasks we need to accomplish:

\begin{enumerate}
\def\labelenumi{\arabic{enumi}.}
\item
  Classify the samples by difficulty
\item
  Identify the attributes of a given word associated with each
  classification
\item
  How hard is the word EERIE?
\item
  Discuss the accuracy of your classification model
\end{enumerate}

This is a scatter plot of mean and standard deviation

\textbf{主要指标的散点图}

\includegraphics{C:/Users/Yupeng Su/OneDrive/桌面/美赛建模/Q3图片/聚类前散点图.png}

第三位要求我们分类,但是数据集中并没有事先给定的难度等级。

所以第一步,我们使用聚类分析,对均值与标准差按距离进行无监督的Kmeans聚类,并拟合出概率分布直方图去直观的进行难度分类

第二步,我们使用随机森林分类模型,对单词属性的五个指标(列出),对第一步聚类所形成的target进行映射,拟合出一个分类器,方便后续对单词的预测。

第三步,我们将EERIE的单词属性做评估,再放入训练好的随机森林模型中做预测,给EERIE这个单词做难度定性。

Accordingly, the model construction and analysis of the third question
can be divided into four steps:

\begin{enumerate}
\def\labelenumi{\arabic{enumi}.}
\item
  Using percentage and included attributes to do cluster analysis

  In this section, we use K-means clustering analysis to cluster the
  mean value and standard deviation according to the distance, and fit
  the probability distribution histogram for difficulty classification
\item
  Using word attributes to map classification by classification model

  We use the random forest classification model to map the five
  indicators of word attributes (EN, TR, GS, YS, CS) and the target
  formed by the clustering in the first step, and fit a classifier to
  facilitate the subsequent word prediction.
\item
  Put EERIE into a classifier for sorting

  We evaluated the word attributes of EERIE, then put them into the
  trained random forest model for prediction, and determined the
  difficulty of the word EERIE.
\item
  Get a representation of the model\textquotesingle s performance on the
  test set
\end{enumerate}

\hypertarget{ux6a21ux578bux539fux7406}{%
\subsubsection{模型原理}\label{ux6a21ux578bux539fux7406}}

K-means clustering is a common unsupervised learning algorithm used to
divide a data set into multiple different groups (clusters). The
similarity of data points within each group is high, while the
similarity between different groups is low. The optimization goal of
K-means algorithm is to minimize the Sum of distance (SSE, Sum of
Squared Errors) between each data point and the clustering center of the
cluster it belongs to. Therefore, the main steps of k-means clustering
include selecting the appropriate K value, initializing the clustering
center, allocating data points to the cluster, calculating the center
point of the cluster, and iterating until convergence is reached.

The principle of random forest classification is the same as that of
random forest regression used in the second question, so I will not
repeat it here.

\hypertarget{ux6a21ux578bux5047ux8bbe}{%
\subsubsection{模型假设}\label{ux6a21ux578bux5047ux8bbe}}

\begin{enumerate}
\def\labelenumi{\arabic{enumi}.}
\item
  Suppose that the difficulty of the word is described only by the
  percentage distribution of the number of reports, independent of other
  factors such as the percentage of difficult patterns chosen.
\item
  Suppose that the percentage distribution of the number of reports is
  close to a normal distribution, which can be well described by means
  and standard deviations alone.
\item
  Suppose that the mean of the distribution of reporting times can
  better reflect the difficulty of words than the standard deviation. In
  K-means analysis, the weight ratio of mean to standard deviation is
  6:5.
\item
  Suppose that noise caused by chance will not obscure the
  characteristics of standard deviation and mean
\end{enumerate}

\hypertarget{ux4e3bux8981ux6a21ux578b}{%
\subsubsection{主要模型}\label{ux4e3bux8981ux6a21ux578b}}

The first step is to construct the clustering model with K-means
clustering

\begin{enumerate}
\def\labelenumi{\arabic{enumi}.}
\item
  The data is normalized and the mean is multiplied by 1.2, making the
  weight ratio of mean to standard deviation 6:5.
\item
  Using elbow method and Silhouette Coefficient, the optimal clustering
  number was determined to be 4(左右各一张图)

  \includegraphics{C:/Users/Yupeng Su/OneDrive/桌面/美赛建模/Q3图片/手肘法.png}
\end{enumerate}

\includegraphics{C:/Users/Yupeng Su/OneDrive/桌面/美赛建模/Q3图片/轮廓系数.png}

\begin{enumerate}
\def\labelenumi{\arabic{enumi}.}
\item
  Here is the clustering result (the asterisk is the clustering center)

  \includegraphics{C:/Users/Yupeng Su/OneDrive/桌面/美赛建模/Q3图片/聚类图像.png}
\item
  The distribution histogram and normal distribution diagram of
  different difficulty are summarized

  \begin{itemize}
  \item
    Simple

    \includegraphics{C:/Users/Yupeng Su/OneDrive/桌面/美赛建模/Q3图片/类别4平均分布(简单).png}
  \item
    Ordinary

    \includegraphics{C:/Users/Yupeng Su/OneDrive/桌面/美赛建模/Q3图片/类别2平均分布(一般).png}
  \item
    Rather Difficult

    \includegraphics{C:/Users/Yupeng Su/OneDrive/桌面/美赛建模/Q3图片/类别1平均分布(较难).png}
  \item
    Difficult

    \includegraphics{C:/Users/Yupeng Su/OneDrive/桌面/美赛建模/Q3图片/类别3平均分布(困难).png}
  \end{itemize}
\end{enumerate}

Second, the random forest classification model is no longer used to
classify word features

\begin{enumerate}
\def\labelenumi{\arabic{enumi}.}
\item
  Adjustment parameter

  The optimal parameters of random forest are determined by grid search
  and learning curve

  \textbf{large-scale learning curve}

  \includegraphics{C:/Users/Yupeng Su/OneDrive/桌面/美赛建模/Q3图片/大纵深学习曲线.png}

  \textbf{Small region learning curve}

  \includegraphics{C:/Users/Yupeng Su/OneDrive/桌面/美赛建模/Q3图片/小范围学习曲线.png}
\end{enumerate}

\begin{longtable}[]{@{}ll@{}}
\toprule\noalign{}
parameter & Value \\
\midrule\noalign{}
\endhead
\bottomrule\noalign{}
\endlastfoot
\texttt{n\_estimators} & 189 \\
\texttt{max\_depth} & 10 \\
\texttt{min\_samples\_leaf} & 6 \\
\texttt{min\_samples\_split} & None \\
\end{longtable}

\begin{enumerate}
\def\labelenumi{\arabic{enumi}.}
\item
  Training random forest classification models

  Classified image

  \includegraphics{C:/Users/Yupeng Su/OneDrive/桌面/美赛建模/Q3图片/随机森林分类图像.png}

  Accuracy for the training set:0.736

  Accuracy for the testing set:0.523
\item
  The importance of the feature
\end{enumerate}

\begin{longtable}[]{@{}ll@{}}
\toprule\noalign{}
Feature & Degree of importance (\%) \\
\midrule\noalign{}
\endhead
\bottomrule\noalign{}
\endlastfoot
entropy & 37.54 \\
text\_rate & 15.58 \\
green\_score & 21.12 \\
yellow\_score & 14.30 \\
conbine\_score & 11.44 \\
\end{longtable}

\begin{enumerate}
\def\labelenumi{\arabic{enumi}.}
\item
  Predict word difficulty(EERIE)

  \textbf{Predicted classification}:(Difficult)

  \textbf{Classification probability}

  \begin{longtable}[]{@{}ll@{}}
  \toprule\noalign{}
  classification & probability \\
  \midrule\noalign{}
  \endhead
  \bottomrule\noalign{}
  \endlastfoot
  Simple & 9.54 \\
  Ordinary & 14.09 \\
  Rather difficult & 13.63 \\
  difficult & 62.73 \\
  \end{longtable}
\end{enumerate}

\end{document}
